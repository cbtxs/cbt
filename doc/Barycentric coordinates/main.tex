% !Mode:: "TeX:UTF-8"
\documentclass[12pt,a4paper]{article}

%%%%%%%%------------------------------------------------------------------------
%%%% 日常所用宏包

%% 控制页边距
% 如果是beamer文档类, 则不用geometry
\makeatletter
\@ifclassloaded{beamer}{}{\usepackage[top=2.5cm, bottom=2.5cm, left=2.5cm, right=2.5cm]{geometry}}
\makeatother

\usepackage{amsthm}
%% 控制项目列表
\usepackage{enumerate}

%% Todo list
\usepackage{enumitem}
\newlist{todolist}{itemize}{2}
\setlist[todolist]{label=$\square$}
\usepackage{pifont}
\newcommand{\cmark}{\ding{51}}%
\newcommand{\xmark}{\ding{55}}%
\newcommand{\done}{\rlap{$\square$}{\raisebox{2pt}{\large\hspace{1pt}\cmark}}%
\hspace{-2.5pt}}
\newcommand{\wontfix}{\rlap{$\square$}{\large\hspace{1pt}\xmark}}


\usepackage{framed}

%% 多栏显示
\usepackage{multicol}

%% 算法环境
\usepackage{algorithm}  
\usepackage{algorithmic} 
\usepackage{float} 

%% 网址引用
\usepackage{url}

%% 控制矩阵行距
\renewcommand\arraystretch{1.4}

%% 粗体
\usepackage{bm}


%% hyperref宏包,生成可定位点击的超链接,并且会生成pdf书签
\makeatletter
\@ifclassloaded{beamer}{
\usepackage{hyperref}
\usepackage{ragged2e} % 对齐
}{
\usepackage[%
    pdfstartview=FitH,%
    CJKbookmarks=true,%
    bookmarks=true,%
    bookmarksnumbered=true,%
    bookmarksopen=true,%
    colorlinks=true,%
    citecolor=blue,%
    linkcolor=blue,%
    anchorcolor=green,%
    urlcolor=blue%
]{hyperref}
}
\makeatother



\makeatletter % 如果是 beamer 不需要下面两个包
\@ifclassloaded{beamer}{
\mode<presentation>
{
} 
}{
%% 控制标题
\usepackage{titlesec}
%% 控制目录
\usepackage{titletoc}
}
\makeatother

%% 控制表格样式
\usepackage{booktabs}

%% 控制字体大小
\usepackage{type1cm}

%% 首行缩进,用\noindent取消某段缩进
\usepackage{indentfirst}

%% 支持彩色文本、底色、文本框等
\usepackage{color,xcolor}

%% AMS LaTeX宏包: http://zzg34b.w3.c361.com/package/maths.htm#amssymb
\usepackage{amsmath,amssymb}
%% 多个图形并排
\usepackage{subfloat}
%%%% 基本插图方法
%% 图形宏包
\usepackage{graphicx}
\newcommand{\red}[1]{\textcolor{red}{#1}}
\newcommand{\blue}[1]{\structure{#1}}
\newcommand{\brown}[1]{\textcolor{brown}{#1}}
\newcommand{\green}[1]{\textcolor{green}{#1}}


%%%% 基本插图方法结束

%%%% pgf/tikz绘图宏包设置
\usepackage{pgf,tikz}
\usetikzlibrary{shapes,automata,snakes,backgrounds,arrows}
\usetikzlibrary{mindmap}
%% 可以直接在latex文档中使用graphviz/dot语言,
%% 也可以用dot2tex工具将dot文件转换成tex文件再include进来
%% \usepackage[shell,pgf,outputdir={docgraphs/}]{dot2texi}
%%%% pgf/tikz设置结束


\makeatletter % 如果是 beamer 不需要下面两个包
\@ifclassloaded{beamer}{

}{
%%%% fancyhdr设置页眉页脚
%% 页眉页脚宏包
\usepackage{fancyhdr}
%% 页眉页脚风格
\pagestyle{plain}
}

%% 有时会出现\headheight too small的warning
\setlength{\headheight}{15pt}

%% 清空当前页眉页脚的默认设置
%\fancyhf{}
%%%% fancyhdr设置结束


\makeatletter % 对 beamer 要重新设置
\@ifclassloaded{beamer}{

}{
%%%% 设置listings宏包用来粘贴源代码
%% 方便粘贴源代码,部分代码高亮功能
\usepackage{listings}

%% 设置listings宏包的一些全局样式
%% 参考http://hi.baidu.com/shawpinlee/blog/item/9ec431cbae28e41cbe09e6e4.html
\lstset{
showstringspaces=false,              %% 设定是否显示代码之间的空格符号
numbers=left,                        %% 在左边显示行号
numberstyle=\tiny,                   %% 设定行号字体的大小
basicstyle=\scriptsize,                    %% 设定字体大小\tiny, \small, \Large等等
keywordstyle=\color{blue!70}, commentstyle=\color{red!50!green!50!blue!50},
                                     %% 关键字高亮
frame=shadowbox,                     %% 给代码加框
rulesepcolor=\color{red!20!green!20!blue!20},
escapechar=`,                        %% 中文逃逸字符,用于中英混排
xleftmargin=2em,xrightmargin=2em, aboveskip=1em,
breaklines,                          %% 这条命令可以让LaTeX自动将长的代码行换行排版
extendedchars=false                  %% 这一条命令可以解决代码跨页时,章节标题,页眉等汉字不显示的问题
}

\usepackage{minted}
\renewcommand{\listingscaption}{Python code} \newminted{python}{
    escapeinside=||,
    mathescape=true,
    numbersep=5pt,
    linenos=true,
    autogobble,
    framesep=3mm} 
}
\makeatother
%%%% listings宏包设置结束


%%%% 附录设置
\makeatletter % 对 beamer 要重新设置
\@ifclassloaded{beamer}{

}{
\usepackage[title,titletoc,header]{appendix}
}
\makeatother
%%%% 附录设置结束





%% 设定行距
\linespread{1}

%% 粗体的小写字母代表向量或向量函数
\newcommand{\bfa}{{\boldsymbol a}}
\newcommand{\bfb}{{\boldsymbol b}}
\newcommand{\bfc}{{\boldsymbol c}}
\newcommand{\bfd}{{\boldsymbol d}}
\newcommand{\bfe}{{\boldsymbol e}}
\newcommand{\bff}{{\boldsymbol f}}
\newcommand{\bfg}{{\boldsymbol g}}
\newcommand{\bfh}{{\boldsymbol h}}
\newcommand{\bfi}{{\boldsymbol i}}
\newcommand{\bfj}{{\boldsymbol j}}
\newcommand{\bfk}{{\boldsymbol k}}
\newcommand{\bfl}{{\boldsymbol l}}
\newcommand{\bfm}{{\boldsymbol m}}
\newcommand{\bfn}{{\boldsymbol n}}
\newcommand{\bfo}{{\boldsymbol o}}
\newcommand{\bfp}{{\boldsymbol p}}
\newcommand{\bfq}{{\boldsymbol q}}
\newcommand{\bfr}{{\boldsymbol r}}
\newcommand{\bfs}{{\boldsymbol s}}
\newcommand{\bft}{{\boldsymbol t}}
\newcommand{\bfu}{{\boldsymbol u}}
\newcommand{\bfv}{{\boldsymbol v}}
\newcommand{\bfw}{{\boldsymbol w}}
\newcommand{\bfx}{{\boldsymbol x}}
\newcommand{\bfy}{{\boldsymbol y}}
\newcommand{\bfz}{{\boldsymbol z}}

\newcommand{\mca}{{\mathcal a}}
\newcommand{\mcb}{{\mathcal b}}
\newcommand{\mcc}{{\mathcal c}}
\newcommand{\mcd}{{\mathcal d}}
\newcommand{\mce}{{\mathcal e}}
\newcommand{\mcf}{{\mathcal f}}
\newcommand{\mcg}{{\mathcal g}}
\newcommand{\mch}{{\mathcal h}}
\newcommand{\mci}{{\mathcal i}}
\newcommand{\mcj}{{\mathcal j}}
\newcommand{\mck}{{\mathcal k}}
\newcommand{\mcl}{{\mathcal l}}
\newcommand{\mcm}{{\mathcal m}}
\newcommand{\mcn}{{\mathcal n}}
\newcommand{\mco}{{\mathcal o}}
\newcommand{\mcp}{{\mathcal p}}
\newcommand{\mcq}{{\mathcal q}}
\newcommand{\mcr}{{\mathcal r}}
\newcommand{\mcs}{{\mathcal s}}
\newcommand{\mct}{{\mathcal t}}
\newcommand{\mcu}{{\mathcal u}}
\newcommand{\mcv}{{\mathcal v}}
\newcommand{\mcw}{{\mathcal w}}
\newcommand{\mcx}{{\mathcal x}}
\newcommand{\mcy}{{\mathcal y}}
\newcommand{\mcz}{{\mathcal z}}

\newcommand{\mra}{{\mathrm a}}
\newcommand{\mrb}{{\mathrm b}}
\newcommand{\mrc}{{\mathrm c}}
\newcommand{\mrd}{{\mathrm d}}
\newcommand{\mre}{{\mathrm e}}
\newcommand{\mrf}{{\mathrm f}}
\newcommand{\mrg}{{\mathrm g}}
\newcommand{\mrh}{{\mathrm h}}
\newcommand{\mri}{{\mathrm i}}
\newcommand{\mrj}{{\mathrm j}}
\newcommand{\mrk}{{\mathrm k}}
\newcommand{\mrl}{{\mathrm l}}
\newcommand{\mrm}{{\mathrm m}}
\newcommand{\mrn}{{\mathrm n}}
\newcommand{\mro}{{\mathrm o}}
\newcommand{\mrp}{{\mathrm p}}
\newcommand{\mrq}{{\mathrm q}}
\newcommand{\mrr}{{\mathrm r}}
\newcommand{\mrs}{{\mathrm s}}
\newcommand{\mrt}{{\mathrm t}}
\newcommand{\mru}{{\mathrm u}}
\newcommand{\mrv}{{\mathrm v}}
\newcommand{\mrw}{{\mathrm w}}
\newcommand{\mrx}{{\mathrm x}}
\newcommand{\mry}{{\mathrm y}}
\newcommand{\mrz}{{\mathrm z}}

%% 粗体的大写字母一般表示矩阵和张量
\newcommand{\bfA}{{\boldsymbol A}}
\newcommand{\bfB}{{\boldsymbol B}}
\newcommand{\bfC}{{\boldsymbol C}}
\newcommand{\bfD}{{\boldsymbol D}}
\newcommand{\bfE}{{\boldsymbol E}}
\newcommand{\bfF}{{\boldsymbol F}}
\newcommand{\bfG}{{\boldsymbol G}}
\newcommand{\bfH}{{\boldsymbol H}}
\newcommand{\bfI}{{\boldsymbol I}}
\newcommand{\bfJ}{{\boldsymbol J}}
\newcommand{\bfK}{{\boldsymbol K}}
\newcommand{\bfL}{{\boldsymbol L}}
\newcommand{\bfM}{{\boldsymbol M}}
\newcommand{\bfN}{{\boldsymbol N}}
\newcommand{\bfO}{{\boldsymbol O}}
\newcommand{\bfP}{{\boldsymbol P}}
\newcommand{\bfQ}{{\boldsymbol Q}}
\newcommand{\bfR}{{\boldsymbol R}}
\newcommand{\bfS}{{\boldsymbol S}}
\newcommand{\bfT}{{\boldsymbol T}}
\newcommand{\bfU}{{\boldsymbol U}}
\newcommand{\bfV}{{\boldsymbol V}}
\newcommand{\bfW}{{\boldsymbol W}}
\newcommand{\bfX}{{\boldsymbol X}}
\newcommand{\bfY}{{\boldsymbol Y}}
\newcommand{\bfZ}{{\boldsymbol Z}}

%% 花体大写字母
\newcommand{\mcA}{{\mathcal A}}
\newcommand{\mcB}{{\mathcal B}}
\newcommand{\mcC}{{\mathcal C}}
\newcommand{\mcD}{{\mathcal D}}
\newcommand{\mcE}{{\mathcal E}}
\newcommand{\mcF}{{\mathcal F}}
\newcommand{\mcG}{{\mathcal G}}
\newcommand{\mcH}{{\mathcal H}}
\newcommand{\mcI}{{\mathcal I}}
\newcommand{\mcJ}{{\mathcal J}}
\newcommand{\mcK}{{\mathcal K}}
\newcommand{\mcL}{{\mathcal L}}
\newcommand{\mcM}{{\mathcal M}}
\newcommand{\mcN}{{\mathcal N}}
\newcommand{\mcO}{{\mathcal O}}
\newcommand{\mcP}{{\mathcal P}}
\newcommand{\mcQ}{{\mathcal Q}}
\newcommand{\mcR}{{\mathcal R}}
\newcommand{\mcS}{{\mathcal S}}
\newcommand{\mcT}{{\mathcal T}}
\newcommand{\mcU}{{\mathcal U}}
\newcommand{\mcV}{{\mathcal V}}
\newcommand{\mcW}{{\mathcal W}}
\newcommand{\mcX}{{\mathcal X}}
\newcommand{\mcY}{{\mathcal Y}}
\newcommand{\mcZ}{{\mathcal Z}}

%% 空心大写字母
\newcommand{\mbA}{{\mathbb A}}
\newcommand{\mbB}{{\mathbb B}}
\newcommand{\mbC}{{\mathbb C}}
\newcommand{\mbD}{{\mathbb D}}
\newcommand{\mbE}{{\mathbb E}}
\newcommand{\mbF}{{\mathbb F}}
\newcommand{\mbG}{{\mathbb G}}
\newcommand{\mbH}{{\mathbb H}}
\newcommand{\mbI}{{\mathbb I}}
\newcommand{\mbJ}{{\mathbb J}}
\newcommand{\mbK}{{\mathbb K}}
\newcommand{\mbL}{{\mathbb L}}
\newcommand{\mbM}{{\mathbb M}}
\newcommand{\mbN}{{\mathbb N}}
\newcommand{\mbO}{{\mathbb O}}
\newcommand{\mbP}{{\mathbb P}}
\newcommand{\mbQ}{{\mathbb Q}}
\newcommand{\mbR}{{\mathbb R}}
\newcommand{\mbS}{{\mathbb S}}
\newcommand{\mbT}{{\mathbb T}}
\newcommand{\mbU}{{\mathbb U}}
\newcommand{\mbV}{{\mathbb V}}
\newcommand{\mbW}{{\mathbb W}}
\newcommand{\mbX}{{\mathbb X}}
\newcommand{\mbY}{{\mathbb Y}}
\newcommand{\mbZ}{{\mathbb Z}}

\newcommand{\mrA}{{\mathrm A}}
\newcommand{\mrB}{{\mathrm B}}
\newcommand{\mrC}{{\mathrm C}}
\newcommand{\mrD}{{\mathrm D}}
\newcommand{\mrE}{{\mathrm E}}
\newcommand{\mrF}{{\mathrm F}}
\newcommand{\mrG}{{\mathrm G}}
\newcommand{\mrH}{{\mathrm H}}
\newcommand{\mrI}{{\mathrm I}}
\newcommand{\mrJ}{{\mathrm J}}
\newcommand{\mrK}{{\mathrm K}}
\newcommand{\mrL}{{\mathrm L}}
\newcommand{\mrM}{{\mathrm M}}
\newcommand{\mrN}{{\mathrm N}}
\newcommand{\mrO}{{\mathrm O}}
\newcommand{\mrP}{{\mathrm P}}
\newcommand{\mrQ}{{\mathrm Q}}
\newcommand{\mrR}{{\mathrm R}}
\newcommand{\mrS}{{\mathrm S}}
\newcommand{\mrT}{{\mathrm T}}
\newcommand{\mrU}{{\mathrm U}}
\newcommand{\mrV}{{\mathrm V}}
\newcommand{\mrW}{{\mathrm W}}
\newcommand{\mrX}{{\mathrm X}}
\newcommand{\mrY}{{\mathrm Y}}
\newcommand{\mrZ}{{\mathrm Z}}


\newcommand{\balpha}{{\bm \alpha}}
\newcommand{\bbeta}{{\bm \beta}}
\newcommand{\bgamma}{{\bm \gamma}}
\newcommand{\bdelta}{{\bm \delta}}
\newcommand{\bepsilon}{{\bm \epsilon}}
\newcommand{\bvarepsilon}{{\bm \varepsilon}}
\newcommand{\bzeta}{{\bm \zeta}}
\newcommand{\bfeta}{{\bm \eta}}
\newcommand{\btheta}{{\bm \theta}}
\newcommand{\biota}{{\bm \iota}}
\newcommand{\bkappa}{{\bm \kappa}}
\newcommand{\blamda}{{\bm \lamda}}
\newcommand{\bmu}{{\bm \mu}}
\newcommand{\bnu}{{\bm \nu}}
\newcommand{\bxi}{{\bm \xi}}
\newcommand{\bomicron}{{\bm \omicron}}
\newcommand{\bpi}{{\bm \pi}}
\newcommand{\brho}{{\bm \rho}}
\newcommand{\bsigma}{{\bm \sigma}}
\newcommand{\btau}{{\bm \tau}}
\newcommand{\bupsilon}{{\bm \upsilon}}
\newcommand{\bphi}{{\bm \phi}}
\newcommand{\bchi}{{\bm \chi}}
\newcommand{\bpsi}{{\bm \psi}}

\newcommand{\rmd}{\,\mathrm d}
\newcommand{\cT}{\mathcal T}
\newcommand{\cF}{\mathcal F}
\newcommand{\cS}{\mathcal S}
\newcommand{\cP}{\mathcal P}
\newcommand{\cM}{\mathcal M}
\newcommand{\cA}{\mathcal A}
\newcommand{\cE}{\mathcal E}
\newcommand{\cB}{\mathcal B}
\newcommand{\cQ}{\mathcal Q}
\newcommand{\cN}{\mathcal N}
\newcommand{\cV}{\mathcal V}
\newcommand{\cW}{\mathcal W}
\newcommand{\bbS}{\mathbb S}
\newcommand{\bbR}{\mathbb R}

%% 算子
\newcommand{\od}{\text{div}}
\newcommand{\os}{\text{span}}
\newcommand{\ot}{\text{tr}}
\newcommand{\norm}[1]{||#1||}
\newcommand{\dof}{\text{dof}}

%%%% 个性设置结束
%%%%%%%%------------------------------------------------------------------------


%%%%%%%%------------------------------------------------------------------------
%%%% bibtex设置

%% 设定参考文献显示风格
% 下面是几种常见的样式
% * plain: 按字母的顺序排列,比较次序为作者、年度和标题
% * unsrt: 样式同plain,只是按照引用的先后排序
% * alpha: 用作者名首字母+年份后两位作标号,以字母顺序排序
% * abbrv: 类似plain,将月份全拼改为缩写,更显紧凑
% * apalike: 美国心理学学会期刊样式, 引用样式 [Tailper and Zang, 2006]

%\makeatletter
%\@ifclassloaded{beamer}{
%\bibliographystyle{apalike}
%}{
%\bibliographystyle{abbrv}
%}
%\makeatother


%%%% bibtex设置结束
%%%%%%%%------------------------------------------------------------------------

%%%%%%%%------------------------------------------------------------------------
%%%% xeCJK相关宏包

\usepackage{xltxtra,fontspec,xunicode}
\usepackage[slantfont, boldfont]{xeCJK} 

\setlength{\parindent}{2em}%中文缩进两个汉字位

%% 针对中文进行断行
\XeTeXlinebreaklocale "zh"             

%% 给予TeX断行一定自由度
\XeTeXlinebreakskip = 0pt plus 1pt minus 0.1pt

%%%% xeCJK设置结束                                       
%%%%%%%%------------------------------------------------------------------------

%%%%%%%%------------------------------------------------------------------------
%%%% xeCJK字体设置

%% 设置中文标点样式,支持quanjiao、banjiao、kaiming等多种方式
\punctstyle{kaiming}                                        
                                                     
%% 设置缺省中文字体
%\setCJKmainfont[BoldFont={Adobe Heiti Std}, ItalicFont={Adobe Kaiti Std}]{Adobe Song Std}   
\setCJKmainfont{SimSun}
%% 设置中文无衬线字体
%\setCJKsansfont[BoldFont={Adobe Heiti Std}]{Adobe Kaiti Std}  
%% 设置等宽字体
%\setCJKmonofont{Adobe Heiti Std}                            

%% 英文衬线字体
\setmainfont{DejaVu Serif}                                  
%% 英文等宽字体
\setmonofont{DejaVu Sans Mono}                              
%% 英文无衬线字体
\setsansfont{DejaVu Sans}                                   

%% 定义新字体
\setCJKfamilyfont{song}{Adobe Song Std}                     
\setCJKfamilyfont{kai}{Adobe Kaiti Std}
\setCJKfamilyfont{hei}{Adobe Heiti Std}
\setCJKfamilyfont{fangsong}{Adobe Fangsong Std}
\setCJKfamilyfont{lisu}{LiSu}
\setCJKfamilyfont{youyuan}{YouYuan}

%% 自定义宋体
\newcommand{\song}{\CJKfamily{song}}                       
%% 自定义楷体
\newcommand{\kai}{\CJKfamily{kai}}                         
%% 自定义黑体
\newcommand{\hei}{\CJKfamily{hei}}                         
%% 自定义仿宋体
\newcommand{\fangsong}{\CJKfamily{fangsong}}               
%% 自定义隶书
\newcommand{\lisu}{\CJKfamily{lisu}}                       
%% 自定义幼圆
\newcommand{\youyuan}{\CJKfamily{youyuan}}                 

%%%% xeCJK字体设置结束
%%%%%%%%------------------------------------------------------------------------

%%%%%%%%------------------------------------------------------------------------
%%%% 一些关于中文文档的重定义
\newcommand{\chntoday}{\number\year\,年\,\number\month\,月\,\number\day\,日}
%% 数学公式定理的重定义

%% 中文破折号,据说来自清华模板
\newcommand{\pozhehao}{\kern0.3ex\rule[0.8ex]{2em}{0.1ex}\kern0.3ex}

\newtheorem{example}{例}                                   
\newtheorem{theorem}{定理}[section]                         
\newtheorem{definition}{定义}
\newtheorem{axiom}{公理}
\newtheorem{property}{性质}
\newtheorem{proposition}{命题}
\newtheorem{lemma}{引理}
\newtheorem{corollary}{推论}
\newtheorem{remark}{注解}
\newtheorem{condition}{条件}
\newtheorem{conclusion}{结论}
\newtheorem{assumption}{假设}

\makeatletter %
\@ifclassloaded{beamer}{

}{
%% 章节等名称重定义
\renewcommand{\contentsname}{目录}     
\renewcommand{\indexname}{索引}
\renewcommand{\listfigurename}{插图目录}
\renewcommand{\listtablename}{表格目录}
\renewcommand{\appendixname}{附录}
\renewcommand{\appendixpagename}{附录}
\renewcommand{\appendixtocname}{附录}
%% 设置chapter、section与subsection的格式
\titleformat{\chapter}{\centering\huge}{第\thechapter{}章}{1em}{\textbf}
\titleformat{\section}{\centering\sihao}{\thesection}{1em}{\textbf}
\titleformat{\subsection}{\xiaosi}{\thesubsection}{1em}{\textbf}
\titleformat{\subsubsection}{\xiaosi}{\thesubsubsection}{1em}{\textbf}

\@ifclassloaded{book}{

}{
\renewcommand{\abstractname}{摘要}
}
}
\makeatother

\renewcommand{\figurename}{图}
\renewcommand{\tablename}{表}

\makeatletter
\@ifclassloaded{book}{
\renewcommand{\bibname}{参考文献}
}{
\renewcommand{\refname}{参考文献} 
}
\makeatother

\floatname{algorithm}{算法}
\renewcommand{\algorithmicrequire}{\textbf{输入:}}
\renewcommand{\algorithmicensure}{\textbf{输出:}}

%%%% 中文重定义结束
%%%%%%%%------------------------------------------------------------------------


\title{重心坐标函数}
\author{陈春雨}
\date{\chntoday}
\begin{document}
\maketitle
\newpage
\section{一维情况}
一维中只用两个点$x_0,x_1$,找两个定义在$[x_0,x_1]$上的线性函数$f_0(x),f_1(x)$,满足:
\begin{align}
\begin{cases}
f_i(x_i)=1&{}\\
f_i(x_j)=0&i\ne j
\end{cases}
\qquad i=0,1
\end{align}
\subsection*{方法一:待定系数法}
假设
$$
f_i(x)=k_0^{(i)}x+k_1^{(i)},\qquad i=0,1
$$
得到方程:
$$
\begin{cases}
k_0^{(0)}x_0+k_1^{(0)}=1\\
k_0^{(0)}x_1+k_1^{(0)}=0
\end{cases},\qquad
\begin{cases}
k_0^{(1)}x_0+k_1^{(1)}=0\\
k_0^{(1)}x_1+k_1^{(1)}=1
\end{cases}
$$
则根据克拉默法则
$$
k_0^{(0)}=\frac{\begin{vmatrix}
                               1&1\\
                               0&1
                              \end{vmatrix}}{\begin{vmatrix}
                                              x_0&1\\
                                              x_1&1
                                             \end{vmatrix}}=\frac{1}{x_0-x_1},\qquad
k_1^{(0)}=\frac{\begin{vmatrix}
                               x_0&1\\
                               x_1&0
                              \end{vmatrix}}{\begin{vmatrix}
                                              x_0&1\\
                                              x_1&1
                                             \end{vmatrix}}=\frac{-x_1}{x_0-x_1}                                             
$$
$$
k_0^{(1)}=\frac{\begin{vmatrix}
                               0&1\\
                               1&1
                              \end{vmatrix}}{\begin{vmatrix}
                                              x_0&1\\
                                              x_1&1
                                             \end{vmatrix}}=\frac{-1}{x0-x1},\qquad
k_1^{(1)}=\frac{\begin{vmatrix}
                               x_0&0\\
                               x_1&1
                              \end{vmatrix}}{\begin{vmatrix}
                                              x_0&1\\
                                              x_1&1
                                             \end{vmatrix}}=\frac{x_0}{x_0-x_1}                                             
$$
这样就得到了两个基函数,且他们的梯度为:
$$
\nabla f_0(x)=k_0^{(0)}\qquad \nabla f_1(x)=k_0^{(1)}
$$
\subsection*{方法二:长度函数}
令$f_0(x)$是$x$到$x_1$的长度与$x_0$到$x_1$的长度之比,$f_1(x)$是$x$到$x_0$的长度与$x_1$到$x_0$的长度之比,所以$f_0,f_1$满足(1)条件,且
$$
\begin{cases}
f_0(x)=\frac{x-x_1}{x_0-x_1}\\
f_1(x)=\frac{x-x_0}{x_1-x_0}
\end{cases}
$$
所以$f_0,f_1$是线性函数,所以$f_0,f_1$是要找的两个函数。
因为$f_0(x)$是$x$到$x_1$的长度与$x_0$到$x_1$的长度之比,所以$f_0$沿着$x_1$到$x_0$的方向时$f_0(x)$函数值增长,所以其梯度方向为$x_1$到$x_0$的方向,又因为线性函数的梯度恒定不变,所以其大小为$$|\frac{f_0(x_0)-f_0(x_1)}{x_0-x_1}|=|\frac{1}{x_0-x_1}|$$\\
所以$f_0(x)$的梯度为:
$$
\nabla f_0(x)=sgn(x_0-x_1)|\frac{1}{x_0-x_1}|=\frac{1}{x_0-x_1}
$$
同理,$f_1(x)$的梯度为:
$$
\nabla f_1(x)=sgn(x_1-x_0)|\frac{1}{x_1-x_0}|=\frac{1}{x_1-x_0}
$$
\section{二维情况}
二维空间中,由三个不共线的点$(x_0,y_0)(x_1,y_1)(x_2,y_2)$找到三个定义在$(x_0,y_0)(x_1,y_1)(x_2,y_2)$围成的三角形区域的线性函数$f_0(x,y),f_1(x,y),f_2(x,y)$,满足:
\begin{align}
\begin{cases}
f_i(x_i,y_i)=1&{}\\
f_i(x_j,y_j)=0&i\ne j
\end{cases}
\qquad i=0,1,2
\end{align}
\subsection*{方法一:待定系数法}
假设
$$
f_i(x,y)=a_ix+b_iy+c_i\qquad i=0,1,2
$$
得到方程:
$$
\begin{cases}
a_0x_0+b_0y_0+c_0=1\\
a_0x_1+b_0y_1+c_0=0\\
a_0x_2+b_0y_2+c_0=0
\end{cases},\qquad
\begin{cases}
a_1x_0+b_1y_0+c_1=0\\
a_1x_1+b_1y_1+c_1=1\\
a_1x_2+b_1y_2+c_1=0
\end{cases}
\begin{cases}
a_2x_0+b_2y_0+c_2=0\\
a_2x_1+b_2y_1+c_2=0\\
a_2x_2+b_2y_2+c_2=1
\end{cases}
$$
设$\lambda=\begin{vmatrix}
                           x_0&y_0&1\\
                           x_1&y_1&1\\
                           x_2&y_2&1
                         \end{vmatrix}$,根据克拉默法则:
$$
a_0=\frac{\begin{vmatrix}
            1&y_0&1\\
            0&y_1&1\\
            0&y_2&1
          \end{vmatrix}}{\lambda},\qquad
b_0=\frac{\begin{vmatrix}
            x_0&1&1\\
            x_1&0&1\\
            x_2&0&1
          \end{vmatrix}}{\lambda},\qquad
c_0=\frac{\begin{vmatrix}
             x_0&y_0&1\\
             x_1&y_1&0\\
             x_2&y_2&0
          \end{vmatrix}}{\lambda}
$$
$$
a_1=\frac{\begin{vmatrix}
            0&y_0&1\\
            1&y_1&1\\
            0&y_2&1
          \end{vmatrix}}{\lambda},\qquad
b_1=\frac{\begin{vmatrix}
            x_0&0&1\\
            x_1&1&1\\
            x_2&0&1
          \end{vmatrix}}{\lambda},\qquad
c_1=\frac{\begin{vmatrix}
             x_0&y_0&0\\
             x_1&y_1&1\\
             x_2&y_2&0
          \end{vmatrix}}{\lambda}
$$
$$
a_2=\frac{\begin{vmatrix}
            0&y_0&1\\
            0&y_1&1\\
            1&y_2&1
          \end{vmatrix}}{\lambda},\qquad
b_2=\frac{\begin{vmatrix}
            x_0&0&1\\
            x_1&0&1\\
            x_2&1&1\\
          \end{vmatrix}}{\lambda},\qquad
c_2=\frac{\begin{vmatrix}
             x_0&y_0&0\\
             x_1&y_1&0\\
             x_2&y_2&1
          \end{vmatrix}}{\lambda}
$$
由此就得到了三个基函数,且他们的梯度为:
$$
\nabla f_0(x,y)=(a_0,b_0),\nabla f_1(x,y)=(a_1,b_1),\nabla f_1(x,y)=(a_1,b_1)
$$
\subsection*{方法二:面积函数}
假设$(x_0,y_0)\rightarrow(x_1,y_1)\rightarrow(x_2,y_2)$是逆时针方向\\
\begin{itemize}
\item 令$f_0(x,y)$表示$[(x,y)(x_1,y_1)(x_2,y_2)]$围成的三角形的面积比上$[(x_0,y_0)(x_1,y_1)(x_2,y_2)]$围成的三角形面积.\\
\item $f_1(x,y)$表示$[(x_0,y_0)(x,y)(x_2,y_2)]$围成的三角形的面积比上$[(x_0,y_0)(x_1,y_1)(x_2,y_2)]$围成的三角形面积.\\
\item $f_2(x,y)$表示$[(x_0,y_0)(x_1,y_1)(x,y)]$围成的三角形的面积比上$[(x_0,y_0)(x_1,y_1)(x_2,y_2)]$围成的三角形面积.\\
\end{itemize}
这时,$f_0(x,y),f_1(x,y),f_2(x,y)$满足条件(2)。\\
用$\tau$来表示总面积,$\tau_i$来表示$f_i(x,y)$对应的面积,由三角形面积公式可得
$$
\tau_0=\frac{1}{2}\begin{vmatrix}
x&y&1\\
x_1&y_1&1\\
x_2&y_2&1
\end{vmatrix},\qquad
\tau_1=\frac{1}{2}\begin{vmatrix}
x_0&y_0&1\\
x&y&1\\
x_2&y_2&1
\end{vmatrix}
$$

$$
\tau_2=\frac{1}{2}\begin{vmatrix}
x_0&y_0&1\\
x_1&y_1&1\\
x&y&1
\end{vmatrix},\qquad
\tau=\frac{1}{2}\begin{vmatrix}
x_0&y_0&1\\
x_1&y_1&1\\
x_2&y_2&1
\end{vmatrix}
$$
由假设可得:
$$
f_0(x,y)=\frac{\tau_0}{\tau},\quad f_1(x,y)=\frac{\tau_1}{\tau},\quad f_2(x,y)=\frac{\tau_2}{\tau}
$$
则$f_0(x,y),f_1(x,y),f_2(x,y)$是线性函数。所以它们是要找的函数。\\
$\qquad$由于$f_0(x,y)$表示的是${(x,y)(x_1,y_1)(x_2,y_2)}$围成的三角形的面积比上${(x_0,y_0)(x_1,y_1)(x_2,y_2)}$围成的三角形面积。所以当$(x,y)$沿着与$\vec{(x_2-x_1,y_2-y_1)}$方向垂直并指向$(x_0,y_0)$的方向,即$(y_1-y_2,x_2-x_1)$方向时,$f_0(x,y)$的函数值增加最快,所以$f_0(x,y)$的梯度方向为$(y_1-y_2,x_2-x_1)$,单位方向为$\vec{n}=\frac{(y_1-y_2,x_2-x_1)}{\sqrt{(y_1-y_2)^2+(x_2-x_1)^2}}$\\
$\qquad$由于$f_0(x,y)$是线性函数,所以梯度不变,假设由$(x_0,y_0)$到对边的垂线的垂足为$(x_3,y_3)$那么垂线长度就是三角形的高,设为$h$,$h=\frac{2\tau}{\sqrt{(y_1-y_2)^2+(x_2-x_1)^2}}$
则$\nabla f_0(x,y)$的大小为$\frac{f_0(x_0,y_0)-f_0(x_3,y_3)}{h}=\frac{\sqrt{(y_1-y_2)^2+(x_2-x_1)^2}}{2\tau}$
$$
\nabla f_0(x,y)=\frac{(y_1-y_2,x_2-x_1)}{\sqrt{(y_1-y_2)^2+(x_2-x_1)^2}}\frac{\sqrt{(y_1-y_2)^2+(x_2-x_1)^2}}{2\tau}=\frac{(y_1-y_2,x_2-x_1)}{2\tau}
$$
同理:
$$
\nabla f_1(x,y)=\frac{(y_2-y_0,x_0-x_2)}{2\tau}
$$
$$
\nabla f_2(x,y)=\frac{(y_0-y_1,x_1-x_0)}{2\tau}
$$
\section{三维情况}
三维空间中,由四个不共面的点$(x_0,y_0,z_0)(x_1,y_1,z_1)(x_2,y_2,z_2)(x_3,y_3,z_3)$确定四个定义域在四个点围成的三棱柱内的线性函数$f_0(x,y,z),f_1(x,y,z),f_2(x,y,z),f_3(x,y,z)$满足:
\begin{align}
\begin{cases}
f_i(x_i)=1&{}\\
f_i(x_j)=0&i\ne j
\end{cases}
\qquad i=0,1,2,3
\end{align}
\subsection*{方法一:待定系数法}设:
$$
f_i(x,y,z)=a_ix+b_iy+c_iz+d_i\qquad i=0,1,2,3
$$
得到方程组:
$$
\begin{cases}
a_0x_0+b_0y_0+c_0z_0+d_0=1\\
a_0x_1+b_0y_1+c_0z_1+d_0=0\\
a_0x_2+b_0y_2+c_0z_2+d_0=0\\
a_0x_3+b_0y_3+c_0z_3+d_0=0
\end{cases},\qquad
\begin{cases}
a_1x_0+b_1y_0+c_1z_0+d_1=0\\
a_1x_1+b_1y_1+c_1z_1+d_1=1\\
a_1x_2+b_1y_2+c_1z_2+d_1=0\\
a_1x_3+b_1y_3+c_1z_3+d_1=0
\end{cases}
$$
$$
\begin{cases}
a_2x_0+b_2y_0+c_2z_0+d_2=0\\
a_2x_1+b_2y_1+c_2z_1+d_2=0\\
a_2x_2+b_2y_2+c_2z_2+d_2=1\\
a_2x_3+b_2y_3+c_2z_3+d_2=0
\end{cases},\qquad
\begin{cases}
a_3x_0+b_3y_0+c_3z_0+d_3=0\\
a_3x_1+b_3y_1+c_3z_1+d_3=0\\
a_3x_2+b_3y_2+c_3z_2+d_3=0\\
a_3x_3+b_3y_3+c_3z_3+d_3=1
\end{cases}
$$
设$\lambda=\begin{vmatrix}
             x_0&y_0&z_0&1\\
             x_1&y_1&z_1&1\\
             x_2&y_2&z_2&1\\
             x_3&y_3&z_3&1
\end{vmatrix}$,根据克拉默法则:
$$
a_0=\frac{\begin{vmatrix}
            1&y_0&z_0&1\\
            0&y_1&z_1&1\\
            0&y_2&z_2&1\\
            0&y_3&z_3&1
          \end{vmatrix}}{\lambda},\quad
b_0=\frac{\begin{vmatrix}
            x_0&1&z_0&1\\
            x_1&0&z_1&1\\
            x_2&0&z_2&1\\
            x_3&0&z_3&1
          \end{vmatrix}}{\lambda},\quad
c_0=\frac{\begin{vmatrix}
            x_0&y_0&1&1\\
            x_1&y_1&0&1\\
            x_2&y_2&0&1\\
            x_3&y_3&0&1
          \end{vmatrix}}{\lambda},\quad
d_0=\frac{\begin{vmatrix}
            x_0&y_0&z_0&1\\
            x_1&y_1&z_1&0\\
            x_2&y_2&z_2&0\\
            x_3&y_3&z_3&0
          \end{vmatrix}}{\lambda}          
$$
$$
a_1=\frac{\begin{vmatrix}
            0&y_0&z_0&1\\
            1&y_1&z_1&1\\
            0&y_2&z_2&1\\
            0&y_3&z_3&1
          \end{vmatrix}}{\lambda},\quad
b_1=\frac{\begin{vmatrix}
            x_0&0&z_0&1\\
            x_1&1&z_1&1\\
            x_2&0&z_2&1\\
            x_3&0&z_3&1
          \end{vmatrix}}{\lambda},\quad
c_1=\frac{\begin{vmatrix}
            x_0&y_0&0&1\\
            x_1&y_1&1&1\\
            x_2&y_2&0&1\\
            x_3&y_3&0&1
          \end{vmatrix}}{\lambda},\quad
d_1=\frac{\begin{vmatrix}
            x_0&y_0&z_0&0\\
            x_1&y_1&z_1&1\\
            x_2&y_2&z_2&0\\
            x_3&y_3&z_3&0
          \end{vmatrix}}{\lambda}          
$$
$$
a_2=\frac{\begin{vmatrix}
            0&y_0&z_0&1\\
            0&y_1&z_1&1\\
            1&y_2&z_2&1\\
            0&y_3&z_3&1
          \end{vmatrix}}{\lambda},\quad
b_2=\frac{\begin{vmatrix}
            x_0&0&z_0&1\\
            x_1&0&z_1&1\\
            x_2&1&z_2&1\\
            x_3&0&z_3&1
          \end{vmatrix}}{\lambda},\quad
c_2=\frac{\begin{vmatrix}
            x_0&y_0&0&1\\
            x_1&y_1&0&1\\
            x_2&y_2&1&1\\
            x_3&y_3&0&1
          \end{vmatrix}}{\lambda},\quad
d_2=\frac{\begin{vmatrix}
            x_0&y_0&z_0&0\\
            x_1&y_1&z_1&0\\
            x_2&y_2&z_2&1\\
            x_3&y_3&z_3&0
          \end{vmatrix}}{\lambda}          
$$
$$
a_3=\frac{\begin{vmatrix}
            0&y_0&z_0&1\\
            0&y_1&z_1&1\\
            0&y_2&z_2&1\\
            1&y_3&z_3&1
          \end{vmatrix}}{\lambda},\quad
b_3=\frac{\begin{vmatrix}
            x_0&0&z_0&1\\
            x_1&0&z_1&1\\
            x_2&0&z_2&1\\
            x_3&1&z_3&1
          \end{vmatrix}}{\lambda},\quad
c_3=\frac{\begin{vmatrix}
            x_0&y_0&0&1\\
            x_1&y_1&0&1\\
            x_2&y_2&0&1\\
            x_3&y_3&1&1
          \end{vmatrix}}{\lambda},\quad
d_3=\frac{\begin{vmatrix}
            x_0&y_0&z_0&0\\
            x_1&y_1&z_1&0\\
            x_2&y_2&z_2&0\\
            x_3&y_3&z_3&1
          \end{vmatrix}}{\lambda}          
$$
这样就得到了四个基函数,且他们的梯度为:
$$
\nabla f_0(x,y)=(a_0,b_0,c_0),\nabla f_1(x,y)=(a_1,b_1,c_1),\nabla f_2(x,y)=(a_2,b_2,c_2),\nabla f_3(x,y)=(a_3,b_3,c_3)
$$
\subsection*{方法二:体积函数}
若$(x_1-x_0,y_1-y_0,z_1-z_0),(x_2-x_0,y_2-y_0,z_2-z_0),(x_3-x_0,y_3-y_0,z_3-z_0)$是右手坐标系。\\
\begin{itemize}
\item 令$f_0(x,y,z)$表示 \{$(x,y,z),(x_1,y_1,z_1),(x_2,y_2,z_2),(x_3,y_3,z_3)$\}围成的三棱锥的体积比上\{$(x_0,y_0,z_0),(x_1,y_1,z_1),(x_2,y_2,z_2),(x_3,y_3,z_3)$\}围成的三棱锥的体积。\\
\item $f_1(x,y,z)$表示\{$(x_0,y_0,z_0),(x,y,z),(x_2,y_2,z_2),(x_3,y_3,z_3)$\}围成的三棱锥的体积比上\{$(x_0,y_0,z_0),(x_1,y_1,z_1),(x_2,y_2,z_2),(x_3,y_3,z_3)$\}围成的三棱锥的体积。\\
\item $f_2(x,y,z)$表示\{$(x_0,y_0,z_0),(x_1,y_1,z_1),(x,y,z),(x_3,y_3,z_3)$\}围成的三棱锥的体积比上\{$(x_0,y_0,z_0),(x_1,y_1,z_1),(x_2,y_2,z_2),(x_3,y_3,z_3)$\}围成的三棱锥的体积。\\
\item $f_3(x,y,z)$表示\{$(x_0,y_0,z_0),(x_1,y_1,z_1),(x_2,y_2,z_2),(x,y,z)$\}围成的三棱锥的体积比上\{$(x_0,y_0,z_0),(x_1,y_1,z_1),(x_2,y_2,z_2),(x_3,y_3,z_3)$\}围成的三棱锥的体积。\\
\end{itemize}

这时,$f_0(x,y,z),f_1(x,y,z),f_2(x,y,z)$满足条件(3)。\\
用$v$来表示总体积,$v_i$来表示$f_i(x,y)$对应的棱锥的体积,由三棱锥体积公式可得:
$$
v=\frac{1}{6}\begin{vmatrix}
x_1&y_1&z_1&1\\
x_2&y_2&z_2&1\\
x_3&y_3&z_3&1\\
x_0&y_0&z_0&1\\
\end{vmatrix},\quad
v_0=\frac{1}{6}\begin{vmatrix}
x_1&y_1&z_1&1\\
x_2&y_2&z_2&1\\
x_3&y_3&z_3&1\\
x&y&z&1\\
\end{vmatrix},\quad
v_1=\frac{1}{6}\begin{vmatrix}
x&y&z&1\\
x_2&y_2&z_2&1\\
x_3&y_3&z_3&1\\
x_0&y_0&z_0&1\\
\end{vmatrix}
$$
$$
v_2=\frac{1}{6}\begin{vmatrix}
x_1&y_1&z_1&1\\
x&y&z&1\\
x_3&y_3&z_3&1\\
x_0&y_0&z_0&1\\
\end{vmatrix},\quad
v_3=\frac{1}{6}\begin{vmatrix}
x_1&y_1&z_1&1\\
x_2&y_2&z_2&1\\
x&y&z&1\\
x_0&y_0&z_0&1\\
\end{vmatrix}
$$
由假设可知:
$$
f_0(x,y,z)=\frac{v_0}{v},\quad f_1(x,y,z)=\frac{v_1}{v}
$$
$$
f_2(x,y,z)=\frac{v_2}{v},\quad f_3(x,y,z)=\frac{v_3}{v}
$$
则$f_0(x,y,z),f_1(x,y,z),f_2(x,y,z),f_3(x,y,z)$是线性函数。所以它们是要找的函数。\\
设$(x_0,y_0,z_0)$的对面为$s_0$,由于$f_0(x,y,z)$是以$s_0$为底,$(x_0,y_0,z_0)$为顶点的立体的体积。所以当$(x,y,z)$沿着与$s_0$垂直且指向$(x_0,y_0,z_0)$的方向时$f_0(x,y,z)$函数值增加最快,所以$f_0(x,y,z)$的梯度的方向是$(x_3-x_1,y_3-y_1,z_3-z_1)\times(x_2-x_1,y_2-y_1,z_2-z_1)$,设$(x_3-x_1,y_3-y_1,z_3-z_1),(x_2-x_1,y_2-y_1,z_2-z_1)$夹角为$\gamma$,则单位方向$\vec{n}$为
$$
\frac{(x_3-x_1,y_3-y_1,z_3-z_1)\times(x_2-x_1,y_2-y_1,z_2-z_1)}{|(x_3-x_1,y_3-y_1,z_3-z_1)||(x_2-x_1,y_2-y_1,z_2-z_1)|sin(\gamma)}
$$
由$(x_0,y_0,z_0)$到$s_0$做垂线,垂足为$(x_4,y_4,z_4)$,设垂线长度为$h$则
$$
h=\frac{6v}{S(s_0)}=\frac{6v}{|(x_3-x_1,y_3-y_1,z_3-z_1)||(x_2-x_1,y_2-y_1,z_2-z_1)|sin(\gamma)}
$$
由于线性函数的梯度处处相等,所以梯度的大小$$l=\frac{f_0(x_0,y_0,z_0)-f_0(x_4,y_4,z_4)}{h}=\frac{1}{h}$$
所以$$
\nabla f_0(x,y,z)=l*\vec{n}=\frac{(x_3-x_1,y_3-y_1,z_3-z_1)\times(x_2-x_1,y_2-y_1,z_2-z_1)}{6v}
$$
同理
$$
\nabla f_1(x,y,z)=\frac{(x_2-x_0,y_2-y_0,z_2-z_0)\times(x_3-x_0,y_3-y_0,z_3-z_0)}{6v}
$$
$$
\nabla f_2(x,y,z)=\frac{(x_3-x_0,y_3-y_0,z_3-z_0)\times(x_1-x_0,y_1-y_0,z_1-z_0)}{6v}
$$
$$
\nabla f_3(x,y,z)=\frac{(x_1-x_0,y_1-y_0,z_1-z_0)\times(x_2-x_0,y_2-y_0,z_2-z_0)}{6v}
$$
























\end{document}
