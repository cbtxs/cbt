
%%%%%%%%------------------------------------------------------------------------
%%%% 日常所用宏包

%% 控制页边距
% 如果是beamer文档类, 则不用geometry
\makeatletter
\@ifclassloaded{beamer}{}{\usepackage[top=2.5cm, bottom=2.5cm, left=2.5cm, right=2.5cm]{geometry}}
\makeatother

%% 控制项目列表
\usepackage{enumerate}

%% 多栏显示
\usepackage{multicol}

%% 算法环境
\usepackage{algorithm}  
\usepackage{algorithmic} 
\usepackage{float} 

%% 网址引用
\usepackage{url}
%
%% 控制矩阵行距
\renewcommand\arraystretch{1.4}

%% hyperref宏包,生成可定位点击的超链接,并且会生成pdf书签
\makeatletter
\@ifclassloaded{beamer}{
\usepackage{hyperref}
\usepackage{ragged2e} % 对齐
}{
\usepackage[%
    pdfstartview=FitH,%
    CJKbookmarks=true,%
    bookmarks=true,%
    bookmarksnumbered=true,%
    bookmarksopen=true,%
    colorlinks=true,%
    citecolor=blue,%
    linkcolor=blue,%
    anchorcolor=green,%
    urlcolor=blue%
]{hyperref}
}
\makeatother



\makeatletter % 如果是 beamer 不需要下面两个包
\@ifclassloaded{beamer}{
\mode<presentation>
{
} 
}{
%% 控制标题
\usepackage{titlesec}
%% 控制目录
\usepackage{titletoc}
}
\makeatother

%% 控制表格样式
\usepackage{booktabs}

%% 控制字体大小
\usepackage{type1cm}

%% 首行缩进,用\noindent取消某段缩进
\usepackage{indentfirst}

%% 支持彩色文本、底色、文本框等
\usepackage{color,xcolor}

%% AMS LaTeX宏包: http://zzg34b.w3.c361.com/package/maths.htm#amssymb
\usepackage{amsmath,amssymb}
%% 多个图形并排
\usepackage{subfig}
%%%% 基本插图方法
%% 图形宏包
\usepackage{graphicx}
\newcommand{\red}[1]{\textcolor{red}{#1}}
\newcommand{\blue}[1]{\structure{#1}}
\newcommand{\brown}[1]{\textcolor{brown}{#1}}
\newcommand{\green}[1]{\textcolor{green}{#1}}


%%%% 基本插图方法结束

%%%% pgf/tikz绘图宏包设置
\usepackage{pgf,tikz}
\usetikzlibrary{shapes,automata,snakes,backgrounds,arrows}
\usetikzlibrary{mindmap}
%% 可以直接在latex文档中使用graphviz/dot语言,
%% 也可以用dot2tex工具将dot文件转换成tex文件再include进来
%% \usepackage[shell,pgf,outputdir={docgraphs/}]{dot2texi}
%%%% pgf/tikz设置结束


\makeatletter % 如果是 beamer 不需要下面两个包
\@ifclassloaded{beamer}{

}{
%%%% fancyhdr设置页眉页脚
%% 页眉页脚宏包
\usepackage{fancyhdr}
%% 页眉页脚风格
\pagestyle{plain}
}

%% 有时会出现\headheight too small的warning
\setlength{\headheight}{15pt}

%% 清空当前页眉页脚的默认设置
%\fancyhf{}
%%%% fancyhdr设置结束


\makeatletter % 对 beamer 要重新设置
\@ifclassloaded{beamer}{

}{
%%%% 设置listings宏包用来粘贴源代码
%% 方便粘贴源代码,部分代码高亮功能
\usepackage{listings}

%% 设置listings宏包的一些全局样式
%% 参考http://hi.baidu.com/shawpinlee/blog/item/9ec431cbae28e41cbe09e6e4.html
\lstset{
showstringspaces=false,              %% 设定是否显示代码之间的空格符号
numbers=left,                        %% 在左边显示行号
numberstyle=\tiny,                   %% 设定行号字体的大小
basicstyle=\footnotesize,                    %% 设定字体大小\tiny, \small, \Large等等
keywordstyle=\color{blue!70}, commentstyle=\color{red!50!green!50!blue!50},
                                     %% 关键字高亮
frame=shadowbox,                     %% 给代码加框
rulesepcolor=\color{red!20!green!20!blue!20},
escapechar=`,                        %% 中文逃逸字符,用于中英混排
xleftmargin=2em,xrightmargin=2em, aboveskip=1em,
breaklines,                          %% 这条命令可以让LaTeX自动将长的代码行换行排版
extendedchars=false                  %% 这一条命令可以解决代码跨页时,章节标题,页眉等汉字不显示的问题
}}
\makeatother
%%%% listings宏包设置结束


%%%% 附录设置
\makeatletter % 对 beamer 要重新设置
\@ifclassloaded{beamer}{

}{
\usepackage[title,titletoc,header]{appendix}
}
\makeatother
%%%% 附录设置结束


%%%% 日常宏包设置结束
%%%%%%%%------------------------------------------------------------------------


%%%%%%%%------------------------------------------------------------------------
%%%% 英文字体设置结束
%% 这里可以加入自己的英文字体设置
%%%%%%%%------------------------------------------------------------------------

%%%%%%%%------------------------------------------------------------------------
%%%% 设置常用字体字号,与MS Word相对应

%% 一号, 1.4倍行距
\newcommand{\yihao}{\fontsize{26pt}{36pt}\selectfont}
%% 二号, 1.25倍行距
\newcommand{\erhao}{\fontsize{22pt}{28pt}\selectfont}
%% 小二, 单倍行距
\newcommand{\xiaoer}{\fontsize{18pt}{18pt}\selectfont}
%% 三号, 1.5倍行距
\newcommand{\sanhao}{\fontsize{16pt}{24pt}\selectfont}
%% 小三, 1.5倍行距
\newcommand{\xiaosan}{\fontsize{15pt}{22pt}\selectfont}
%% 四号, 1.5倍行距
\newcommand{\sihao}{\fontsize{14pt}{21pt}\selectfont}
%% 半四, 1.5倍行距
\newcommand{\bansi}{\fontsize{13pt}{19.5pt}\selectfont}
%% 小四, 1.5倍行距
\newcommand{\xiaosi}{\fontsize{12pt}{18pt}\selectfont}
%% 大五, 单倍行距
\newcommand{\dawu}{\fontsize{11pt}{11pt}\selectfont}
%% 五号, 单倍行距
\newcommand{\wuhao}{\fontsize{10.5pt}{10.5pt}\selectfont}
%%%%%%%%------------------------------------------------------------------------


%% 设定段间距
\setlength{\parskip}{0.5\baselineskip}

%% 设定行距
\linespread{1}


%% 设定正文字体大小
% \renewcommand{\normalsize}{\sihao}

%制作水印
\RequirePackage{draftcopy}
\draftcopyName{XTUMESH}{100}
\draftcopySetGrey{0.90}
\draftcopyPageTransform{40 rotate}
\draftcopyPageX{350}
\draftcopyPageY{80}

%%%% 个性设置结束
%%%%%%%%------------------------------------------------------------------------


%%%%%%%%------------------------------------------------------------------------
%%%% bibtex设置

%% 设定参考文献显示风格
% 下面是几种常见的样式
% * plain: 按字母的顺序排列,比较次序为作者、年度和标题
% * unsrt: 样式同plain,只是按照引用的先后排序
% * alpha: 用作者名首字母+年份后两位作标号,以字母顺序排序
% * abbrv: 类似plain,将月份全拼改为缩写,更显紧凑
% * apalike: 美国心理学学会期刊样式, 引用样式 [Tailper and Zang, 2006]

\makeatletter
\@ifclassloaded{beamer}{
\bibliographystyle{apalike}
}{
\bibliographystyle{unsrt}
}
\makeatother


%%%% bibtex设置结束
%%%%%%%%------------------------------------------------------------------------
