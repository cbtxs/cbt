% !Mode:: "TeX:UTF-8"
\documentclass[12pt,a4paper]{article}
\input{en_preamble.tex}
\input{xecjk_preamble.tex}

\title{ 五点差分法求解椭圆型偏微分方程}
\author{ 陈春雨}
\date{\chntoday}
\begin{document}
\maketitle
\newpage
\section{问题}

\begin{align*}
-\Delta u &=\cos3x\sin\pi y,\quad (x,y)\in G=(0,\pi)\times(0,1) \\
u_x(0,y) &=u_x(\pi,y)=0,\quad 0<y<1,\\
u(x,0) &=u(x,1)=0,\quad 0<x<\pi,\\
\end{align*}
精确解为:$u=(9+\pi^2)^{-1}\cos3x\sin\pi y$
\subsection{ 离散}
对$x,y$的范围等距离散得到
$$
\{x_0,x_1,...x_n\}
\{y_0,y_1,...y_m\}
$$
步长分别为$\frac{\pi}{n},\frac{1}{n}$,用$u_{ij}$代表$u(x_i,y_j)$
\begin{figure}[H]
\centering
\includegraphics[scale=0.15]{./figures/Figure_1.jpg}
\end{figure}

对于上下两个边界上的点,$u_{ij}$满足:
$$
u_{ij}=0
$$
对于左右两个边界的点满足:
\begin{equation}
\frac{\partial u}{\partial x} (x_i,y_j)=0
\end{equation}
对于内部的点满足:
\begin{equation}\frac{\partial^2 u}{\partial x^2}(x_i,y_j)+\frac{\partial^2 u}{\partial y^2}(x_i,y_j)=-\cos3x_i\sin\pi y_j
\end{equation}

由泰勒公式可知:
\begin{align*}
\frac{\partial u}{\partial x} (x_i,y_j)&=\frac{u_{i+1j}-u_{ij}}{h_1}+o(h_1)\\
\frac{\partial u}{\partial x} (x_i,y_j)&=\frac{u_{ij}-u_{i-1j}}{h_1}+o(h_1)\\
\frac{\partial^2 u}{\partial x^2}(x_i,y_j)&=\frac{u_{i+1,j}+u_{i-1,j}-2u_{i,j}}{h_1^2}+o(h_1^2)\\
\frac{\partial^2 u}{\partial y^2}(x_i,y_j)&=\frac{u_{i,j+1}+u_{i,j-1}-2u_{i,j}}{h_2^2}+o(h_2^2)\\
\end{align*}
对(1)进行离散,得到:
$$
u_{i+1j}-u_{ij}=o(h_1)
$$
或:
$$
u_{i-1j}-u_{ij}=o(h_1)
$$
对(2)进行离散,得到:
$$
\frac{u_{i+1,j}+u_{i-1,j}-2u_{i,j}}{h_1^2}+\frac{u_{i,j+1}+u_{i,j-1}-2u_{i,j}}{h_2^2}=-cos3x_isin\pi y_j+o(h_1^2)+o(h_2^2)
$$
综上可得:
\begin{equation}
\left\{
\begin{aligned}
u_{ij}&=0\qquad \qquad(j=0\quad or \quad j=n)\\
u_{i+1j}-u_{ij}&=o(h_1) \qquad (i=0)\\
u_{i-1j}-u_{ij}&=o(h_1) \qquad( i=n)\\
\frac{u_{i+1,j}+u_{i-1,j}-2u_{i,j}}{h_1^2}+\frac{u_{i,j+1}+u_{i,j-1}-2u_{i,j}}{h_2^2}&=-cos3x_isin\pi y_j+o(h_1^2)+o(h_2^2)\qquad (other)
\end{aligned}
\end{equation}

设$U$为$(n+1)^2$维向量,$U(i\times(n+1)+j)=u_{ij}$,则可以得到方程:
$$
AU=b+e
$$
其中$b,e$是$(n+1)^2$维向量,$b$中没有高阶无穷小量,$e$中全是高阶无穷小量且最高阶为$o(h_1)$.所以精确解为:
$$
U=A^{-1}b+A^{-1}e
$$
而数值解为:
$$
U_s=A^{-1}b
$$
所以误差为:$A^{-1}e$,所以以无穷范数做误差的话,误差的阶数是1阶的。

\subsection{计算}
经过编程计算以后,得到:

\begin{tabular}{l|l|l}
\hline

剖分数 & 误差 & 阶\\
\hline
4 & 0.11731442534924674 &{}\\
\hline
8 & 0.047303607146040476 &{1.3}\\
\hline
16 & 0.019117740374945326 & {1.3}\\
\hline
32 & 0.008457203854343821 & {1.1}\\
\hline
64 & 0.003969881130677012 & {1.1}\\
\hline
\end{tabular}












































\end{document}
