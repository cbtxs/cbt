% !Mode:: "TeX:UTF-8"
\documentclass[12pt,a4paper]{article}
\input{en_preamble.tex}
\input{xecjk_preamble.tex}


\title{不规则图形到规则图形的映射}
\author{陈春雨}
\date{\chntoday}
\begin{document}
\maketitle
\newpage
\section{二维图形的映射}
\subsection{ 任意三角形到直角三角形的映射}
假设一个三角形$A$的三个顶点是$P_1,P_2,P_3,$其中$P_i=(x_i,y_i)$,将其映射为顶点为$(0,0),(0,1),(1,0)$的直角三角形$B$。

对于任意$P=(x,y)\in A$,$P-P_1$可以被$P_2-P_1,.P_3-P_1$线性表出,设
$$
P-P_1=a(P_2-P_1)+b(P_3-P_1)
$$
可以做线性映射$f$:
$$
(x,y)\to (a,b)
$$
\begin{figure}[H]
\centering
\includegraphics[scale=0.15]{./figures/Figure_1.jpg}
\end{figure}
任取$ P=(x,y)\in A$,连接$P_1,P$并延长交$P_2P_3$于$P^{'}=(x^{'},y^{'})$则
$$
a(x^{'},y^{'})\ge a(x,y) \ge 0,\quad b(x^{'},y^{'})\ge b(x,y) \ge 0
$$
设$P^{'}P_3$的长度是$P_2P_3$的$k$倍,则
\begin{align*}
P^{'}-P_1&=P_2-P_1+k(P_3-P_2)\\
&=k(P_3-P_1)+(1-k)(P_2-P_1)
\end{align*}
所以
$$a(x^{'},y^{'})=1-k,\quad b(x^{'},y^{'})=k$$
所以$$a(x,y)+b(x,y)\le 1$$
因此$f$把$A$映射到了$B$,而对于任意$(a,b)$属于$B$,可以找到$(x,y)=a(P_2-P_1)+b(P_3-P_1)$属于$A$,所以$f$是满射,由线性表出的唯一性可知$f$是一个单射,所以$f$是$A$到$B$的一一映射。

由$P-P_1=a(P_2-P_1)+b(P_3-P_1)$可得:
$$
\begin{cases}
x-x_1=a(x_2-x_1)+b(x_3-x_1)\\
y-y_1=a(y_2-P_1)+b(y_3-y_1)
\end{cases}
$$
显然,$f$是一个线性映射,解线性方程组即可得到$(a,b)$.

\subsection{ 任意四边形到正方形的映射}
假设一个四边形$A$的四个顶点是$P_1,P_2,P_3,P_4$其中$P_i=(x_i,y_i)$,将其映射为顶点为$Q_1=(0,0),Q_2=(0,1),Q_3=(1,0),Q_4=(1,1)$的正方形$B$。
\begin{figure}[H]
\centering
\includegraphics[scale=0.15]{./figures/Figure_2.jpg}
\end{figure}
若使用1.1中的线性映射,将$P_1,P_2,P_3$映射为$Q_1,Q_2,Q_3$,则$P_4$不一定映射到$Q_4$,(事实上,只有$A$是平行四边形的时候才会使$P_4$映射到$Q_4$),所以要使用新的映射。

我们发现任取正方形中一点$Q=(\lambda,k)$,经过$Q$做$y$轴平行线,分别交上下边于$Q^{'},Q^{''}$
\begin{figure}[H]
\centering
\includegraphics[scale=0.12]{./figures/Figure_3.jpg}
\end{figure}
则

$$
\frac{l(Q^{'}-Q_3)}{l(Q_4-Q_3)} =\frac{l(Q^{''}-Q_1)}{l(Q_2-Q_1)}=\lambda,\quad \frac{l(Q-Q^{''})}{l(Q^{'}-Q^{''})}=k  
$$

这样就定义了$Q$的坐标,那么同样的方法在一般四边形中是否适用呢?
\begin{definition}
$l(\alpha)=\alpha$的长度
\end{definition}

\begin{definition}
假设一个四边形$A$的四个顶点是$P_1,P_2,P_3,P_4$,$P^{'},P^{''}$分别在$P_1P_2,P_3P_4$上,且
$$
\frac{l(P^{'}-P_3)}{l(P_4-P_3)} =\frac{l(P^{''}-P_1)}{l(P_2-P_1)}=\lambda
$$
则称$P^{'}P^{''}$为等比线,$\lambda$为等比线的比
\end{definition}

\begin{theorem}
对于任意四边形中的任意一点,都有唯一一条等比线经过它。
\end{theorem}


证明:

存在性:假设一个四边形$A$的四个顶点是$P_1,P_2,P_3,P_4$,将$P_1P_2,P_3P_4$向两边无线延长,任取四边形内部一点$P$,作经过$P$的直线,分别交$P_1P_2,P_3P_4$与$P^{'},P^{''}$,作关于直线斜率$k$的函数$g$:
$$
g(k)=\frac{l(P^{'}-P_3)}{l(P_4-P_3)} -\frac{l(P^{''}-P_1)}{l(P_2-P_1)}
$$
则$g(k)$是一个连续函数,设$P^{''}=P_1$时,直线斜率为$k_1$,$P^{''}=P_2$时,直线斜率为$k_2$,
则
$$
g(k_1)>0,\quad g(k_2)<0
$$
所以存在一个$k$使得$g(k)=0$,即存在一条过$P$的直线满足
$$
\frac{l(P^{'}-P_3)}{l(P_4-P_3)} =\frac{l(P^{''}-P_1)}{l(P_2-P_1)}
$$所以这是一条等比线。

唯一性:若存在两条等比线$Q^{'}Q^{''}$, $P^{'}P^{''}$过$P$,假设$Q^{'}$在$P^{'}$左边,则$Q^{''}$在$P^{''}$右边,那么两条线至少有一条不是等比线,所以每个点只有一条等比线经过。证毕||

由定理1.1可知,四边形内任意一个点$P=(x,y)$有唯一一条等比线$P^{'}P^{''}$经过,设这个等比线的比为$\lambda$,
$$
\frac{l(P-P^{''})}{l(P^{'}-P^{''})}=k
$$
作映射$f$:
$$
(x,y)\to (\lambda,k)
$$
由于$0\le \lambda,k \le 1$,所以$f$把四边形映射到了正方形内.

任取$0\le \lambda,k \le 1$,都存在四边形上比为$\lambda$的等比线$P^{'}P^{''}$,在这条线上定存在一点$P$,满足
$$
\frac{l(P-P^{''})}{l(P^{'}-P^{''})}=k
$$
根据$f$的定义可知:
$$
f(P)=(\lambda,k)
$$
所以$f$是一个满射,由于等比线的唯一性可知$f$是一个单射,所以$f$是$A$到$B$的一一映射。

用待定系数法,假设$P=(x,y)$:
$$
f(P)=(\lambda,k)
$$
则:
\begin{align*}
P-P_1&=P-P^{''}+P^{''}-P_1\\
&=k(P^{'}-P{''})+\lambda(P_2-P_1)\\
&=k(P^{'}-P_3)+k(P_3-P_1)+k(P_1-P^{''})+\lambda(P_2-P_1)\\
&=\lambda k(P_4-P_3)+k(P_3-P_1)+\lambda(1-k)(P_2+P_1)
\end{align*}
所以得到方程组:
$$
\begin{cases}
x-x_1&=\lambda k(x_4-x_3)+k(x_3-x_1)+\lambda(1-k)(x_2+x_1)\\
y-y_1&=\lambda k(y_4-y_3)+k(y_3-y_1)+\lambda(1-k)(y_2+y_1)
\end{cases}
$$
由于$f$是一一映射,所以对于任意$(x,y)$,都有解。解方程组即可得到$(\lambda,k)$
\section{ 三维}
\subsection{ 任意三棱柱到直角三棱柱的映射}
假设一个三棱柱的六个顶点为:$P_1,P_2,P_3,P_4,P_5,P_6$,其中$P_i=(x_i,y_i,z_i)$将其映射为顶点为将其映射为顶点为$Q_1=(0,0,0),Q_2=(0,1,0),Q_3=(1,0,0),Q_4=(0,0,1),Q_5=(0,1,1),Q_6=(1,0,1)$的直角三棱柱。

\begin{figure}[H]
\centering
\includegraphics[scale=0.15]{./figures/Figure_4.jpg}
\end{figure}

\begin{definition}
假设一个三棱柱的六个顶点为:$P_1,P_2,P_3,P_4,P_5,P_6$,在三条棱上分别取三个点$P^{'},P^{''},P^{''}$,若:
$$
\frac{l(P^{'}-P_4)}{l(P_1-P_4)} =\frac{l(P^{''}-P_5)}{l(P_2-P_5)} =\frac{l(P^{'''}-P_6)}{l(P_4-P_6)} =b
$$
则称$P^{'},P^{''},P^{''}$组成的三角形为三棱柱的等比面,$b$成为等比面的比。\\
\end{definition}
\begin{theorem}
对任意三棱柱,取内部任意一点,都有唯一一个等比面经过。
\end{theorem}
证明:存在性:假设一个三棱柱的六个顶点为:$P_1,P_2,P_3,P_4,P_5,P_6$,取内部一点$P=(x,y,z)$,在$P_3P_6,P_2P_5$上分别取两个点$P^{''},P^{''}$,令:
$$
\frac{l(P^{''}-P_5)}{l(P_2-P_5)} =\frac{l(P^{'''}-P_6)}{l(P_4-P_6)}
$$
$P,P^{''},P^{'''}$组成一个平面,交$P_1P_4$于$P^{'}$	,设:
$$
g=\frac{l(P^{'}-P_4)}{l(P_1-P_4)} -\frac{l(P^{''}-P_5)}{l(P_2-P_5)}
$$
则$g$与平面$P^{'},P^{''},P^{''}$与三个坐标轴夹角角度$\theta_1,\theta_2,\theta_3$有关,显然他是连续的,设$P^{''}=P_5$时,夹角为$t_{11},t_{21},t_{31}$,$P^{''}=P_2$时,夹角为$t_{12},t_{22},t_{32}$,则:
$$
g(t_{11},t_{21},t_{31})\le 0,g(t_{12},t_{22},t_{32})\ge 0
$$
所以存在$\theta_1,\theta_2,\theta_3$,使:
$$
g(\theta_1,\theta_2,\theta_3)=0
$$
即存在$P^{'}$,使:
$$
\frac{l(P^{'}-P_4)}{l(P_1-P_4)} =\frac{l(P^{''}-P_5)}{l(P_2-P_5)} =\frac{l(P^{'''}-P_6)}{l(P_4-P_6)}
$$
即$P^{'},P^{''},P^{'''}$是一个等比面。

唯一性,与定理1.1类似。||

由定理2.1可知,对于三棱柱内任意一点$P$,在三条棱上有三个点$P^{'},P^{''},P^{''}$使$P$在三个点组成的三角形内。且
$$
\frac{l(P^{'}-P_4)}{l(P_1-P_4)} =\frac{l(P^{''}-P_5)}{l(P_2-P_5)} =\frac{l(P^{'''}-P_6)}{l(P_4-P_6)}=b
$$
所以存在$\lambda,k$使$P=\lambda P_{'}+k P_{''}+(1-\lambda-k)P_{'''}$,$0\le b \le 1,0\le \lambda+k\le 1$
作映射$f$:
$$
(x,y,z)\to (\lambda,k,b)
$$
则$f$把三棱柱映射为直角三棱柱,$\forall (\lambda,k,b)$属于直角三棱柱,原三棱柱存在比为$b$的等比面$P^{'},P^{''},P^{''}$,令$P=\lambda P_{'}+k P_{''}+(1-\lambda-k)P_{'''}$,则由$f$的定义可知:
$$
f(P)=(\lambda,k,b)
$$
所以$f$是一个满射,由定理2.1可知,$f$是一个单射,所以$f$是一个一一映射。

综上,若$f(P)=(\lambda,k,b)$,可以得到如下关系:
\begin{align*}
P&=\lambda P^{'}+k P^{''}+(1-\lambda+k)P^{''}\\
&=\lambda (b P_4+(1-b)P_1)+k(b P_5+(1-b)P_2)(1-\lambda+k)(b P_6+(1-b)P_3)
\end{align*}
得到方程组:
$$
\begin{cases}
x=\lambda (b x_4+(1-b)x_1)+k(b x_5+(1-b)x_2)(1-\lambda+k)(b x_6+(1-b)x_3)\\
y=\lambda (b y_4+(1-b)y_1)+k(b y_5+(1-b)y_2)(1-\lambda+k)(b y_6+(1-b)y_3)\\
z=\lambda (b z_4+(1-b)z_1)+k(b z_5+(1-b)z_2)(1-\lambda+k)(b z_6+(1-b)z_3)\\
\end{cases}
$$
由定理2.1可以方程组必有解,求解方程组即可得到$(\lambda,k,b)$。


































\end{document}
