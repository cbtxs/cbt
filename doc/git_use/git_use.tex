% !Mode:: "TeX:UTF-8"
\documentclass{article}
\input{en_preamble.tex}
\input{xecjk_preamble.tex}
\begin{document}

\title{git基本使用}
\author{陈春雨}
\maketitle
\section{基本操作}
\begin{listing}[H]
    \begin{pythoncode}
    #在 github/gitlab 的仓库中复制仓库链接 URL, 使用下面代码将仓库克隆下来.
    git clone URL



    #查看当前仓库状态:(显示上次上传后仓库的改变)
    git status
    
    git diff                 #查看仓库未暂存的改动
    
    git diff --cached        #查看仓库已经暂存的改动



    #上传文件
    git pull                 #将远端仓库的更新拉下来

    git add .                #将仓库所有的改变, 暂存起来

    git commit -m'update'    #提交暂存的文件并添加注释 update)

    git pull                 #再次检查是否与远端一致

    git push (#上传)



    #在仓库中添加 .gitignort 文件
    *.[a0]                   #表示不上传 .a 或 .o 结尾的文件

    *~                       #表示不上传以 ~ 结尾的文件

    *.pdf                    #表示不上传pdf, 以此类推,不需要上传的就可以添加进来

    !main.pdf                #表示 main.pdf 文件可以上传上去

    doc/**/*.pdf             #表示 doc 文件夹里的 .pdf 文件不上传
    
    
    
    #远程仓库
    git remote -v            #查看远程仓库的命名和链接
    
    git remote rename        #给远程仓库重命名
    
    git remote add origin URL       #添加一个名为 origin 的网址为 URL 的远端仓库
    \end{pythoncode}
\end{listing}


\cite{fem_2010}
\bibliographystyle{abbrv}
\bibliography{ref}
\end{document}
