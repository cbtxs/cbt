% !Mode:: "TeX:UTF-8"
\documentclass{article}
\input{en_preamble.tex}
\input{xecjk_preamble.tex}
\begin{document}
\title{二维分片多项式空间}
\author{陈春雨}
\date{\chntoday}
\maketitle
\section{分片线性多项式空间}
给定三角形区域 $I = (x_0, y_0),(x_1,y_1),(x_2,y_2)$, 其上的线性多项式空间定义如下:
\begin{equation}
    P_1(I) = \{v: v(x) = c_0 + c_1 x + c_2 y, (x,y)\in I, c_0, c_1 ,c_2\in \mathbb R\},
\end{equation}
基函数的选择:重心坐标函数(节点基函数)\\

用$\lambda_0$表示在$(x_0,y_0)$处为1,在$(x_1,y_1),(x_2,y_2)$处为0的一次函数,$\lambda_1$表示在$(x_1,y_1)$处为1,在$(x_0,y_0),(x_2,y_2)$处为0的一次函数,$\lambda_2$表示在$(x_2,y_2)$处为1,在$(x_1,y_1),(x_0,y_0)$处为0的一次函数.\\

$\forall(x,y)\in I$ 假设
\begin{align*}
I_0 = (x, y),(x_1,y_1),(x_2,y_2)\\
I_1 = (x_0, y_0),(x,y),(x_2,y_2)\\
I_2 = (x_0, y_0),(x_1,y_1),(x,y)\\
\end{align*}\\
则由重心坐标函数的性质可知:
\begin{equation}
    \lambda_0(x,y) = \frac{s(I_0)}{s(I)}, \lambda_1(x,y) = \frac{s(I_1)}{s(I)}. \lambda_2(x,y) = \frac{s(I_2)}{s(I)}.
\end{equation}

\section{$k$ 次多项式空间 $P_k(I)$}

给定区间 $I = [x_0, x_1]$, 其上的线性多项式空间定义如下:
\begin{equation}
    P_k(I) = \{v: v(x) = c_{00} + c_{10} x + c_{20} x^2 + \cdots + c_{k0} x^k + c_{01} y+ \cdots +c{0k}y^k, (x,y)\in I, c_{00}, c_{10},
    \cdots c_{k0} \in \mathbb R\},
\end{equation}

$I$ 上的 $ k\geq 1 $ 次基函数共有 

\[
n_{dof} = \frac{(k+1)(k+2)}{2},
\]

其计算公式如下:

\[
\phi_{m,n,r} = \frac{1}{m!n!r!}\prod_{l_0 = 0}^{m - 1}
(k\lambda_0 - l_0) \prod_{l_1 = 0}^{n-1}(k\lambda_1 -
l_1)\prod_{l_2 = 0}^{r-1}(k\lambda_2 -
l_2).
\]

其中 $ m\geq 0$, $ n\geq 0 $,$ r\geq 0 $, 且 $ m+n+r=k $, 这里规定:

\[
 \prod_{l_i=0}^{-1}(k\lambda_i - l_i) := 1,\quad i=0, 1, 2
\]

$k$ 次基函数的面向数组的计算
构造向量: 

\[
P = ( \frac{1}{0!},  \frac{1}{1!}, \frac{1}{2!}, \cdots, \frac{1}{k!})
\]

构造矩阵: 

\[
A :=
\begin{pmatrix}
1  &  1  &  1 \\
k\lambda_0 & k\lambda_1  & k\lambda_2\\
k\lambda_0 - 1 & k\lambda_1 - 1 & k\lambda_2 - 1\\
\vdots & \vdots  & \vdots\\
k\lambda_0 - (k - 1) & k\lambda_1 - (k - 1) & k\lambda_2 - (k - 1)
\end{pmatrix}
\]

对 $A$ 的每一列做累乘运算, 并左乘由 $P$ 形成的对角矩阵, 得矩阵:

\[
B = \mathrm{diag}(P)
\begin{pmatrix}
1 & 1 & 1\\
\lambda_0 & \lambda_1 & \lambda_2\\
\prod_{l=0}^{1}(k\lambda_0 - l) & \prod_{l=0}^{1}(k\lambda_1 - l) & \prod_{l=0}^{1}(k\lambda_2 - l)\\
\vdots & \vdots & \vdots\\
\prod_{l=0}^{k-1}(k\lambda_0 - l) & \prod_{l=0}^{k-1}(k\lambda_1 - l)  & \prod_{l=0}^{k-1}(k\lambda_2 - l)
\end{pmatrix}
\]

易知, 只需从 $B$ 的每一列中各选择一项相乘(要求三项次数之和为 $k$,
其中取法共有

\[
n_{dof} = \frac{(k+1)(k+2)}{2}
\]

构造指标矩阵:
\[
I = \begin{pmatrix}
k  & 0 & 0\\ k-1 & 0 & 1 \\ k-1 & 1 & 0 \\ k-2 & 0 & 2 \\ \vdots & \vdots \\ 0 & k & 0
\end{pmatrix}
\]
则第 $i$ 个 $k$ 次基函数可写成如下形式
\[
\phi_i = B_{m,0}B_{n,1}B_{r,2}, 
\]
其中 $ m = I_{i, 0}, n = I_{i, 1}  r = I_{i, 2}$, 并且 $ m + n + r = k$.

\begin{align*}
    \nabla \prod_{j = 0}^{m - 1} (k\lambda_0 - j)
    = & k\sum_{j=0}^{m - 1}\prod_{0\le l \le m-1, l\not= j}(k\lambda_0 -
    l)\nabla \lambda_0,\\
    \nabla \prod_{j = 0}^{n - 1} (k\lambda_1 - j)
    = & k\sum_{j=0}^{n - 1}\prod_{0\le l \le n-1, l\not= j}(k\lambda_1 -
    l)\nabla \lambda_1,\\
    \nabla \prod_{j = 0}^{r - 1} (k\lambda_2 - j)
    = & k\sum_{j=0}^{r - 1}\prod_{0\le l \le r-1, l\not= j}(k\lambda_2 -
    l)\nabla \lambda_2,\\
\end{align*}
\begin{align*}
\bfD^0 = 
\begin{pmatrix}
k & k\lambda_0 & \cdots & k\lambda_0 \\
k\lambda_0 - 1 & k & \cdots & k\lambda_0 - 1 \\
\vdots & \vdots & \ddots & \vdots \\
k\lambda_0 - (k-1) & k\lambda_0 - (k-1) & \cdots & k 
\end{pmatrix},\\ 
\bfD^1 = 
\begin{pmatrix}
k & k\lambda_1 & \cdots & k\lambda_1 \\
k\lambda_1 - 1 & k & \cdots & k\lambda_1 - 1 \\
\vdots & \vdots & \ddots & \vdots \\
k\lambda_1 - (k-1) & k\lambda_1 - (k-1) & \cdots & k 
\end{pmatrix},\\ 
\bfD^1 = 
\begin{pmatrix}
k & k\lambda_2 & \cdots & k\lambda_2 \\
k\lambda_2 - 1 & k & \cdots & k\lambda_2 - 1 \\
\vdots & \vdots & \ddots & \vdots \\
k\lambda_2 - (k-1) & k\lambda_2 - (k-1) & \cdots & k 
\end{pmatrix},\\
\end{align*}
把 $\bfD^0$,$\bfD^1$ 和 $\bfD^2$ 的每一列沿行的方向做累乘运算,然后取它们的下三角矩阵,
最后把下三角矩阵的每一行再求和,即可得到矩阵 $\bfB$ 的每一列各个元素的求导后系
数. 可得到矩阵 $\bfD$,其元素定义为 
$$
\bfD_{i,j} = \sum_{m=0}^j\prod_{k=0}^j D^i_{k, m},\quad 0 \le i \le 2,
, 0 \le j \le k-1.
$$
最后,可以用如下的方式来计算 $\bfB$ 的梯度:
\begin{equation*}
\begin{aligned}
\nabla \bfB = & \mathrm{diag}(\bfP)
\begin{pmatrix}
0 & 0 & 0\\
\bfD_{0,0} \nabla \lambda_0 & 
\bfD_{1,0} \nabla \lambda_1 &
\bfD_{2,0} \nabla \lambda_2\\
\vdots & \vdots\\
\bfD_{0, k-1} \nabla \lambda_0 &
\bfD_{1, k-1} \nabla \lambda_1 & 
\bfD_{2, k-1} \nabla \lambda_2 
\end{pmatrix}\\
= & \mathrm{diag}(\bfP)
\begin{pmatrix}
\mathbf 0\\
\bfD
\end{pmatrix}
\begin{pmatrix}
\nabla \lambda_0 \\
 & \nabla \lambda_1 \\
 & & \nabla \lambda_2\\
\end{pmatrix}\\
= & \bfF 
\begin{pmatrix}
\nabla \lambda_0 \\
 & \nabla \lambda_1 \\
 & & \nabla \lambda_2\\
\end{pmatrix},
\end{aligned}
\end{equation*}
其中
\begin{equation}\label{eq:F}
    \bfF = \mathrm{diag}(\bfP)
\begin{pmatrix} 
    \mathbf 0\\ \bfD
\end{pmatrix}.
\end{equation}



\end{document}
